% !TEX root = ../debate_casefile.tex


\section{Big data} \index{Big data}


\subsection{Smart cities} \index{Big data!Smart cities}

No longer a thing of the future. Data collection for the purpose of meassuring the state of altercations in public locations are good for effective (less intrusive) policing. 

Examples:
\begin{itemize}
	\item Songdo, South Korea
	\item IBM Control Center, Rio de Janeiro
	\item Google in Toronto
	\item Stratumseind Eindhoven, Netherlands
\end{itemize}

Becasue of the \textbf{Personal Data Protection Act} people should be notified in advance of data collection and the purpose should be specified – but in Stratumseind, as in many other “smart cities”, this is not the case.
The argument given is that these measures keep track of \textbf{crowds, not individuals}.

“We often get that comment – ‘Big brother is watching you’ – but I prefer to say, ‘Big brother is helping you’. We want safe nightlife, but not a soldier on every street corner.”

City traffic sensors pick up your phone’s wifi signal even if you are not connected to the wifi network. The trackers register your MAC addressThe city council wants to know how often people visit Enschede, and what their routes and preferred spots are. 

Only those who mine the small print will discover that the app creates “personal mobility profiles”, and that the collected personal data belongs to the company Mobidot.

Most cases however, \textbf{cities don't know what's being collected} by companies.


\subsection{Anonymizing data}

It's often argued that personal ID gets scrambled and lost between the massive amounts of other people (anonymizing through pseudonyms)

But \textbf{pseudonymised personal data is still personal data}. “The process is not irreversible if the source file is stored,“Moreover, if you build personal profiles and act on them by targeting an individual person, you are processing personal data and need to comply with data protection law. 

\subsection{Ownership of data} \index{Big data!Ownership}

Who owns data that is collected in a public space? Because public authorities are increasingly outsourcing tasks to private companies, most municipalities don't bother setting controls for managment.

When contracts are made, companies dictate the terms, and cities say they can’t share the contracts because it contains “competition-sensitive information”.

Ultimately what happens is that \textbf{A smart city is a privatised city}



% !TEX root = ../debate_casefile.tex

\section{Populism}

Populism is a political philosophy supporting the rights and power of the people in their \textbf{struggle against a privileged} elite. Critics of populism have described it as a political approach that seeks to disrupt the existing social order by solidifying and mobilizing the animosity of the "commoner" against "elites" and the "establishment".

Populists can \textbf{fall anywhere} on the traditional left–right political spectrum of politics and often portray both bourgeois capitalists and socialist organizers as unfairly dominating the political sphere. The term has also been used as \textbf{a label for new parties} whose classifications are unclear
\begin{itemize}
	\item Jeremy Corbin (right wing)
	\item Bernie Sanders (left wing)
	\item Hugo Chavez (left)
\end{itemize}



\subsection{Critisism}

The terms "populist" and "populism" as pejoratives against their opponents. Such a view sees populism as demagogy, merely \textbf{appearing to empathize} with the public through rhetoric or unrealistic proposals in order to increase appeal across the political spectrum.

In developing economies, these tend to be misused for gathering attention of people uninformed voters and those unhappy with the current political system.

Be aware of a \textbf{distinction}. Populism and authoritarism has historically gone hand in hand (Nazis or Venezuela), as their framing rellies on ``rising and changing the status quo for what they believe is their (better) world view''. This speech \textbf{doesn't guarantee the continuity} of democracy or its institutions, which is troublesome to investors or allies.

\subsection{Policies} \index{Latin America!Populism}
Populism has been an important force in Latin American political history, where many charismatic leaders have emerged since the beginning of the 20th century, as the paramountcy of agrarian oligarchies had been dislocated by the onset of industrial capitalism, allowing for the emergence of an industrial bourgeoisie and the activation of an urban working class.

Examples of this
\begin{itemize}
	\item Venezuela
	\begin{itemize}
		\item Nationalization of oil
	\end{itemize}
	\item Mexico
	\begin{itemize}
	  	\item Nationalization of oil (1974), now open.
	  	\item Government appropriation of land (\textit{ejidos}) to give to farmers ('The land is owned by those that work it')
	  		\subitem This lead to less government control of rural Mexico, and less services like education or policing (source of \textbf{narcos}) \index{Latin America!War on drugs} \index{Drugs!War on drugs!Mexico}
    \end{itemize}  
\end{itemize}





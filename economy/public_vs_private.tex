% !TEX root = ../debate_casefile.tex

\section{Publicly traded company} \label{sec:publicly_traded_company}


On the market, a company can be publicly tr

An \textbf{IPO} stands for a `Initial Public Offering', which means that's the value a company when it goes public, it offers to buyers.

This can be done by small, young companies looking for capital \textbf{to expand} (such as Facebook, GoPro, Tesla), or by large a private companies looking to be publicly traded.

Why go puiblic:
\begin{itemize}
	\item You get access to more money
	\item You can use that money to acquire smaller players in the same market
	\item You're currently growing, which is a sign that you'll keep growing. Public investors would be knocking at the doors.
	\item 
\end{itemize}

There's \textbf{risks} to being a publicly traded company.
\begin{itemize}
	\item For investor, it is tough to predict what the stock will do as there's little historical data.
	\item For company, it now will have to report to all their NET evaluation, comprised of: 
	\begin{itemize}
		\item \textit{growth}, the cahnge of value
		\item \textit{revenue}, the income from sales/production.
		\item \textit{operatign costs} like salaries (OPEX)
		\item \textit{capital costs} like properties (CAPEX)
	\end{itemize}
	\item 
\end{itemize}

\subsection{Death spiral} \index{Bankruptcy!Death spiral}

When a publicly traded company goes into a death spiral, all future predictions show that the company will go \textbf{bankrupt}

A company is likely to be on a death spiral because of a combination of either of these:
\begin{itemize}
	\item Getting \textit{flanked} by other competitors with better prodcuts/services at a more compettitve prices.
	\item The market chaning around you so that your product/service is no longer relevant
	\item Changing of strategy/structure/key memebers which makes the organization loose legitimacy or effectivness at operating
	\item Layoffs, clossing divisions or selling of assets with the hope of \textit{bouncing back}.
\end{itemize}


\section{Bankruptcy} \index{Bankruptcy}

Is a legal proceeding involving a person or business that is \textbf{unable to repay} outstanding debts. Its usually filled by the \textit{debtors}, though less comonly by the \textit{creditors}.

All of the debtor's assets are meassured and evaluated so that they can be used to repay \textbf{a portion} of the outstanding debt.

Bankruptcy offers an individual or business a chance to start fresh by forgiving debts that simply cannot be paid, while \textbf{offering creditors a chance to obtain some repayment} based on the individual's or business' assets available for liquidation.

In theory, the ability to file for bankruptcy can benefit an overall economy by giving persons and businesses \textbf{a second chance to gain access to consumer credit} and by providing creditors with a measure of debt repayment.

Upon the successful completion of bankruptcy proceedings, the debtor \textbf{is relieved of the debt obligations} incurred prior to filing for bankruptcy.

\subsection{Chapters of bankruptcy} \index{United States!Bankruptcy}
In the US, theres multiple chapters of bankrupcty, the most popular ones being:
\begin{itemize}
	\item[7]For individuals or buisnesses with few or no assets. Theyre forced to \textbf{liquidate} personal valuable assets such as heirlooms, house, vehicles, stocks or bonds. If they posses no valuable  assets, then they \textbf{don't repay} their unsecured debt.
	\item[11] Mostly used by buisnesses to \textbf{reorganize} to later become profitable. It allows the chance to create plans for profitability, cut costs and find new ways to increase revenue. It allows the company to continue conducting its \textbf{daily operations without interruption}, while working on a debt repayment plan under the court's supervision.
	\item[13] Usually used by individuals who make too much money to qualify for Chapter 7. It allows individuals and businesses to \textbf{lower debt repayment} to create workable plans. In exchange for repaying their creditors, the courts allow these debtors to keep all of their property.
	\item[15] Is designed to make legal proceedings of \textbf{international bankruptcies} more predictable and fair for debtors and creditors. It also tries to protect the value of the debtor's assets and, if possible, financially rescue the business.
\end{itemize}

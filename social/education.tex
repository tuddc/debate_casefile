% !TEX root = ../debate_casefile.tex

\section{Education} \index{Education}

\subsection{Learning process} \index{Education!Bad education}
When a student is presented with knowledge they can have any of these reactions:
\begin{itemize}
	\item They think they know it
	\item They don't pay utmost attention
	\item They don't recognize how what was presented differs from their own pre-conceptions
	\item They don't learn anything new
	\item They get \textbf{more comfident} in the ideas they were thinking before
\end{itemize}

\subsection{Pos-modern education}

\index{Education!Good education}
A good future proof education system would

\begin{itemize}
	\item Ask the student of their current world view
	\item Activly present them with incorrect statements that make them question their own understanding of the topic
	\item Show the accurate representation of the facts, so that they \textbf{rebuild their concepts} based on understanding misconceptions.
\end{itemize}

The process is very personalized, and interactive websites or charter schools should be able to fit these new ways of education. \index{Education!Charter schools}


\subsection{Charter schools} \index{Education!Charter schools}



\subsection{Developing countries} \index{Latin America!Education}



\begin{quote}
	\textit{If I can't directly produce value from what you did, then it's of no use to me.}
\end{quote}

Try to develop professionally in an environment where those in power are too selfish and uneducated to use its privileged position, and those below are barely making ends meet.

The group of people that can \textbf{experience} the correlation of ``higher education leads to a better standards of living'' is quite narrow.

Thus, you get an underclass that don't believe in the reforming power of education and hard work, easily fall pray to \textit{shortcuts} to improve their lives; at best, populist measures from weak political institutions looking for re-election or at worst, a personal economy reliant on drug trafficking.





\section{International education} \label{sec:international_education}

10 years ago,  10\% of students weren't Dutch. Now 22\% are international.

Numbers are good for international Rankines
\begin{itemize}
	\item Importiung brains.
	\item Good for the university.
	\item Good for the economy of the Netherlands.
\end{itemize}

The dean of TU Delft says that \textit{We need to have some control}.

Some questions pop-up: Is it legal to set quotas?



	\subsection{Internationalization comprimises qulity of education} 

	Without the quality of policies:
	\begin{itemize}
		\item Number of students
		\item Open without borders.
	\end{itemize}


	There's structural issues with the university.
		The bigger issues are about managing it. TU Delft is a public university, so they handle this burden by raising the price for students away from the EU.

			You can compensate with scolarships.

		Talent has no nationality and deserves good education. Budgets are limited.
		
		Identity is something that isn't acquired. If you have a majority of cultural, there's no guarantee of identity, in the core should there be a \textit{Delft Identity}.

		The \textit{Delft way} exists. The standard is maintained as a mixture of publishing and working. The spirit of engineering should come from wherever. The university wants you because you can contribute to it.

		Every profession needs to give lectures. It's a pre-requisite.


	\subsection{Actors}
	
	\begin{itemize}
		\item Dutch students that can't speak English well.
		\item International students that want to study in TU Delft.
		\item The city that can't house the number of students.
	\end{itemize}
			

	\subsection{Nationality as criterion for admission}

		It  reinforces the idea of \textit{birth lottery}. 
		It becomes an extra thing to consider into CVs. This goes against \textbf{values} of science and dutch society.

		What happens with quota. We should select quality. Education is about improving society.

			[What about Dutch students. Are you going to accept them all?]
			[What about Europeans that come to liberate pressure of the national education]

		Morally it sends a message, which might create a slippery slope.
		
		2/3rd of the university is payed by public money (both Dutch and from the EU). If we give a minimum quota to Dutch students, that would ensure the public money is used into the population.

			On the long run, the Netherlands will benefit from foreigners studying in the Netherlands which \textbf{stay in the country}, as the nature of the country.


	\subsection{Should the University invest in Integration} 

	An easy answer is yes, but a more nuanced would be to:

	\begin{itemize}
		\item Enforce Dutch courses or integration (not economical)
		\item The university already invests enough with the OWEE and the IP, but the burden should be to the student societies.
	\end{itemize}
		
	Go and integrate! not just in your own culture. 


	\subsubsection{Language}

	Language is used to share ideas, but also it's used to convey/retain culture.

	Each field's identification to language isn't the same (Dutch studies, Law and Engineering have different linguistic needs).

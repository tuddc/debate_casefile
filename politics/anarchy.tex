% !TEX root = ../debate_casefile.tex

\section{Anarchy} \index{Anarchy}

It's the condition of a society, entity, group of people, or single person to reject hierarchy.

Anarchy could be linked to a desire for freedom. 

\subsection{Peaceful anarchy}

See, Leo Tolstoy, a Russian farmer turned writer from the 1900's who labeled himself as a \textit{Christian anarchist}. Believed that novels as tools for social change and that structures of power such as governments should be eliminated \textbf{by their own course} and natural social development. 

Examples of these:
\begin{itemize}
    \item Usage of \textit{civil disobedience} by Gandhi
\end{itemize}

The opposite of this movement would be \textbf{violent anarchy}.
Movements such as these have only been successful in remote places in the world, or small communities, where the levels of influence are minimal.

When a violent anarchist movement removes a head of state, or dissolves governmental institutions, a game of \textit{action and reaction} takes place among those who are in the immediate spheres of power trying to fill a vacuum.

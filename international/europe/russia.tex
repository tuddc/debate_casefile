% !TEX root = ../../debate_casefile.tex

\section{Russia} \index{Russia}

Elections in 12 Days. Putin going for 3rd time as president.

An enigmatic coutnry. The larges contry in the world.

\textbf{Putin's russia - by Prof. Evert van der Zweerde (March 6th 2018)}



\subsection{3 Remarks} \label{sub:3_remarks}

`I love russia'. The language, people, culture, food. The people are great, despite its dark sides. `Loving russia doesn't mean you \textbf{dont} love the rest of the world'

People are very critical to russia. Namely: Syria, Domestic politics and Election meddling. Being `for ' or `against' russia is not what we're for. As academics we try to make sense of the world we live in, withoug \textit{judsging} ourselves.

The approach of the speech is about using philosophical, using accurate facts and history to back them up.
`Why do we even hold elections in Russia anyway', but then you could go as far as to ask `why do countries even have elections'. \textbf{Democracy} hsa beeen the standard for the 90+\% of countries after WW2 (Even NK claims to be democratic)


\subsection{Vladimir Putin} \label{sub:vladimir_putin}

Depending on who you ask, he could be said to be a new \textbf{tsar}, a new \textbf{stallin}, new \textbf{hitler}, or an excellent statesman who is what Russia of today needs.

Putin is sober, unlke wat our concepts of what russians are.




\subsection{Power and politics}

\textit{Reality} consints of entities with \textbf{powers to..}
\begin{itemize}
	\item to go
	\item to eat
	\item to order
	\item to kill
\end{itemize}

Theses are \textbf{potentialities} (P1). These powers interact with eachoter, beingin one person or between many people. (self conflict) or among institutions, goverment, organization (horizontal interplay).

The interaction of power implies the \textbf{possibility} of conflict, not reality. This is the \textit{political}, the ubiquitos possibilit of real (lethal) conflict. This can happen within one person and / or groups.


Politics, means about \textbf{power over..} This is the vertical excertion of  of power. This implies an \textit{assymetrical relation}.
This relation implies a level of \textbf{acceptence of authority} \index{Power!Authority}.
The acceptance of authority can always be \textbf{withrdawn} (disobedience, civil war). Every good ruler should be afraid of the eventuallity of the power.

Theres a higher level of power, namely as \textbf{soverignity}. This is consider as (P3).

\subsection{Democracy and Legitimacy} 

Democratic politics is about dealing with power in a democratic way. Its acceptance of constellations of P1, under a P2, and eventualy reaching a P3.

The useage of low level powers, gives legitemacy to the higher powers.

\begin{enumerate}
	\item democratic legitemcay. If there's a historical/cultural awareness of your society
	\item governcance legimacy. Majors tend to have more legitimcay as they directly solve problems of cities, unlike senators.
	\item Institituinal legitimcay. Consider how we have always done things in a way (follwoing traditions.)
\end{enumerate}


\subsection{Ideology} 

\begin{enumerate}
	\item Politcal: liberalism, anarchism, fasism,, socialism
	\item Legitimazing ideology.
	\item Hegemoic. When we \textbf{normalize} it to the point of it being part of our confort zone.
\end{enumerate}
	

In any political context, theres ideloges. There usually connected to poliitcial powers. This is linked to the \textbf{powr to convice people} (this is a form of horizontal, P1). 

Note that political parties, by the word party, implies a division of power encomassed in the form of dcision of society.
Any functional division of power, requres a funtional consellation (\textit{regime}) among each partition of society.

Example:
Communism (I1) becalme I2 after the Bolsheviks sized 1918. The Soviet ideolgy (I3) became hegemnic.


\section{West vs Russia}

\begin{table}[H]
	\centering
	\caption{west vs russia}\label{tab:west_russia}
	%\resizebox{\textwidth}{!}
\end{table}


\subsection{Geography and Demography}

Russian Federation is the largest countyu in the world. 17M km$^2$. While Canada has 10M km$^2$.

Its dictrubtion of population shows that the majority live in the european side. Theres plenty of \textbf{migration} from siderbrai into the european side (getting progressively more empty).

Its a federal reppublic comprised of 83 \textit{subjects} which have some ``autonomos regions'' with their own territory and languagues.

\subsubsection{Problems}

Its geographically \textbf{vulnerable}
\begin{itemize}
	\item large terrotory, not a lot of people (economic)
	\item Flat, and open on all sides (little agriculture, prone to invasion)
	\item Little acces to sea (fear of encrirlemnet > Crimea)
\end{itemize}

Socially
\begin{itemize}
	\item Afraid of being overrun from asian population into siberia.
	\item It had breakway tendency in the 1990s, Siberia at some ont had its own currcny, until it go its shit together. It \textbf{centraized} itself with political powers
	``Why did Siberia join Russia instead of selling directly to asia?'' Russia used its soverign state (P3) headed by the Kremlin (I2).
\end{itemize}


Many people consider that the only way have a stable Russia is by a \textit{united Russia} (I3)


\subsection{Economy}

At the 12th place of the GDP.

It doubled between 1990 and 2005. By 2014 it already tripled from the 1990. This means that peopl who were young have seen tehir country state \textbf{develop greatly}. This only gives more \textbf{legitimacy} the governance of Russia.

\subsubsection{Oil dependecen}

Historically, from the 1990's its GDP has always been dependen tonthe price of oil.

60\% of the expors from russia is in mineral producs. and 75\% are hydrocarbons.

Theyre trying to move away from oil dependecny, but \textbf{corruption} is hard to tackle.
It's heavily tied to the economic (P1), the political (P2) and the power protected bu a soverign state (P3).

\subsubsection{Sanctions} 

A blessing in disgauise.

The sancitons pushed many russian people to work and develop their production industries for food (more in provinces than in cities).

This doesnt mean that sanctions aren't effective, but to the groups of people awho live away from the european Russia, the sanctions have not had negative effects. They have plenty of resources, but have historically not been motivated to change their economic model that rellies on mineral resoiurces.


\subsubsection{Rule of law}

The quality of the excercion of law has improved, in comparison with the 90's. Far more accountability is seen when civil prosecution. This doesn't mean that the state law is fair, as power disparities and corruption easily bend the rule of law in the courts.

This increases the difficulty of middle class people from improving the quality of life.

Russia traditionally has beautiful laws, and regulation, but people ignore them regardless.

The older generations, typically have come to accept the \textit{status quo}, as a way of the politics and law to being the way they are.

The only cases prosecuted, are to smaller cases in small provinces, considered to be forms of \textbf{political theater}.



\subsection{Modern History} 

In the 1900's

Despite its progressive ideoligy (I1) the soviet revime was plitically reactionary when it comes to civil rigehes and livbertittes. It was cultirall conservative (L3).
I t naver calimed to be a communist society. It did clami to contruct socilamim in one contry on the way to a global communism and it pretend t obe guidedby a marxist-Leninist ideology. 

The history of russia is a history of human tragedy
\begin{itemize}
	\item Revoution 1905
	\item WW1
	\item Feb Rev 1917
	\item Oktob rev 1917
	\item Red terror and Civil war
	\item Holdomor 1932
	\item Stallin purges
	\item WW2 and stallingrad
\end{itemize}


\subsection{Putins Russia} 
Russia is a country with a relative authoritarian state, but not totalitarian. It's far more free than the USSR ever was.

Its considerad an authoritarian oligachy witn a 25 ish number of busines tycoons with Puti as the \textbf{chief manager and arbitrer}. He is an independtent (candidate) that doesnt directly get the dirt of the buisness tycoons. The \textbf{white tsar}.


\subsubsection{Why elections} 
BEcause it proviedes a legitimace. Their team want a 70\% of votation with a 70\% representation which means that theres a \textbf{hard representation} of close to 50\% of the population.

\subsubsection{Whats expected}
Putin is afer a new balance of \textbf{internal} power.

War talk is for external offensive. If they work on 

On the external, compared to Trump, he's likely to be a far more rational character in the international scene, which would only increase his legitimacy inside the country.


\subsubsection{Crimea}

Georpahycally, Crimea si very cvaluable. (no isolation)

NATO and EU action on the border might have ben mistaken as a provocation. Russia wants a buffer zone between them and Russia.

Its stratigially one of the few accesible ports for Russia, specially for deploying forces (u-boats).

From the internal point of view, of Russians and Crimeans, they themselves believed to be part of russia. They consider it to be \textbf{rightful}.

It used to be part of turkey, but it became part of 


\subsubsection{Young generation}

The first group of peopl that havent had the memories of the USSR will start voting en-mass.

The 1990's were times of terrible life. Infrastructre and safety has improved. The older people have accepted the current rule, as they have memories of a far worse country.

Both on the big cities, and smaller provinces, the younger generation have the notion of change, but it would take time.


\subsubsection{US Election Intervention} \index{United States!Russia}

The current Russian govennt is looking to destabilize goverments aboad. 

NATO and EU have historically provoked and humiliated Russia, to the point in which they want to have the same kindof 

Russia, like any country wants:
\begin{itemize}
	\item Buffer zones
	\item Military strength
	\item Negotiation power
\end{itemize}

Once the heghemony of the US, disappears, the Russian sentiment of not wanting to be humiliated and playing on an even ground wit the rest of the world, justifies them from the meassures they take.


US has a longer history of destabilizing both allies and enemies all around the world for its own conviniecen

\begin{itemize}
	\item Latin america
\end{itemize}

	




